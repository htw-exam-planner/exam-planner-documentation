%%%%%%%%%%%%%%%%%%%%%%%%% Makros %%%%%%%%%%%%%%%%%%%%%%%%%%%%%%%%%%%%%%%%%%%%%%%

%Semanktik: \testReport{Datum}{Tester}{Testfall}{Ergebnis}{Bewertung}
\newcommand{\testReport}[5]{
  \paragraph{Test am #1, durchgeführt von #2}
  #3
  \subparagraph{Tatsächliches Ergebnis:} #4
  \subparagraph{Bewertung:} #5
}

%Semanktik: \testCase{Gegenstand}{Testfall}{Testdaten}{Rahmenbedingungen}{Erwartetes Ergebnis}
\newcommand{\testCase}[5]{
  \subparagraph{Testgegenstand:} #1
  \subparagraph{Testfall:} #2
  \subparagraph{Testdaten:} #3
  \subparagraph{Rahmenbedingungen:} #4
  \subparagraph{Erwartetes Ergebnis:} #5
}

%%%%%%%%%%%%%% Testfälle %%%%%%%%%%%%%%%%%%%%%%%%%%%%%%%%%%%%%%%%%%%%%%%%%%%%%%%
\newcommand{\testGenerate}{
  \testCase{Teilsystem Prüfungstermine verwalten}
    {Termine und Gruppen generieren}
    {Startdatum: 9.7.2018, Anzahl Gruppen: 10}
    {}
    {15 freie Termine an den Wochentagen ab 9.7., 10 Gruppen}
}

\newcommand{\testGenerateNotMonday}{
  \testCase{Teilsystem Prüfungstermine verwalten}
    {Termine und Gruppen generieren}
    {Startdatum: 10.7.2018, Anzahl Gruppen: 10}
    {}
    {Warnhinweis, da 10.7. kein Montag ist}
}

\newcommand{\testGenerateZeroGroups}{
  \testCase{Teilsystem Prüfungstermine verwalten}
    {Termine und Gruppen generieren}
    {Startdatum: 9.7.2018, Anzahl Gruppen: 0}
    {}
    {Warnhinweis, da 0 Gruppen unzulässig sind}
}

\newcommand{\testFreeDeactivated}{
  \testCase{Teilsystem Prüfungstermine verwalten}
    {Termin als frei markieren}
    {deaktivierter Termin 9.7.2018 aus Testdatensatz}
    {Termin ist deaktiviert}
    {Termin 9.7. hat Zustand frei, ansonsten keine Änderung}
}

\newcommand{\testFreeBooked}{
  \testCase{Teilsystem Prüfungstermine verwalten}
    {Termin als frei markieren}
    {Termin 11.7.2018, gebucht von Gruppe 1 aus Testdatensatz}
    {Termin ist gebucht}
    {Termin 9.7. hat Zustand frei, die Buchung ist nicht mehr vorhanden}
}

\newcommand{\testDeactivateFree}{
  \testCase{Teilsystem Prüfungstermine verwalten}
    {Termin deaktivieren}
    {Termin 10.7. aus Testdatensatz}
    {Termin ist frei}
    {Termin ist deaktiviert, ansonsten keine Änderung}
}

\newcommand{\testDeactivateReserved}{
\testCase{Teilsystem Prüfungstermine verwalten}
  {Termin deaktivieren}
  {Termin 12. 7. aus Testdatensatz}
  {Termin ist reserviert}
  {Termin ist deaktiviert, die Reservierung ist nicht mehr vorhanden}
}

\newcommand{\testSetTimeWindow}{
\testCase{Teilsystem Prüfungstermine verwalten}
  {Zeitfenster bearbeiten}
  {Termin 13.7. aus Testdatensatz, Zeitfenster 08:00-12:00}
  {}
  {Termin mit neuem Zeitfenster 08:00-12:00}
}

\newcommand{\testSetTimeWindowInvalid}{
\testCase{Teilsystem Prüfungstermine verwalten}
  {Zeitfenster bearbeiten}
  {Termin 13.7. aus Testdatensatz, Zeitfenster 12:00-08:00}
  {}
  {Fehlermeldung}
}

\newcommand{\testSetNote}{
\testCase{Teilsystem Prüfungstermine verwalten}
  {Bemerkung bearbeiten}
  {Termin 13.7. aus Testdatensatz, Bemerkung "`Hier könnte Ihre Werbung stehen"'}
  {}
  {Termin hat neue Bemerkung "`Hier könnte Ihre Werbung stehen"'}
}

\newcommand{\testSetBookingRoomTime}{
\testCase{Teilsystem Prüfungstermine verwalten}
  {Raum und Endzeit einer Buchung festlegen}
  {Termin 16.7. aus Testdatensatz, Endzeit 15:00, Raum Z146b}
  {Termin ist durch Gruppe 3 gebucht}
  {Termin mit Endzeit und Raum wie festgelegt}
}

\newcommand{\testSetBookingRoomTimeNotBooked}{
\testCase{Teilsystem Prüfungstermine verwalten}
  {Raum und Endzeit einer Buchung festlegen}
  {Termin 13.7. aus Testdatensatz}
  {Termin ist nicht gebucht}
  {Fehlermeldung, da Termin nicht gebucht}
}

\newcommand{\testShowAppointments}{
\testCase{Teilsystem Termine anzeigen}
  {Termine anzeigen}
  {}
  {Termine wurden bereits generiert}
  {Übersicht der Termine}
}

\newcommand{\testViewGroups}{
\testCase{Teilsystem Gruppen verwalten}
  {Gruppen anzeigen}
  {}
  {Gruppen wurden bereits generiert}
  {Übersicht der Termine}
}

\newcommand{\testCreateGroup}{
\testCase{Teilsystem Gruppen verwalten}
  {Gruppe anlegen}
  {}
  {Gruppen des Testdatensatzes vorhanden}
  {Neue Gruppe mit Nummer 8 angelegt}
}

\newcommand{\testDeleteGroup}{
\testCase{Teilsystem Gruppen verwalten}
  {Gruppe löschen}
  {Gruppe 3}
  {Gruppe 3 hat Termin 16.7. gebucht}
  {Gruppe ist nicht mehr vorhanden, Buchung am 16.7. nicht mehr vorhanden}
}

\newcommand{\testLogInGroup}{
\testCase{Teilsystem Termine anzeigen und buchen}
  {Als Gruppe anmelden}
  {Gruppe 6}
  {}
  {Terminübersicht}
}

\newcommand{\testReserveDeactivated}{
\testCase{Teilsystem Termine anzeigen und buchen}
  {Termin reservieren}
  {Termin 9.7. aus Testdatensatz}
  {Termin ist deaktiviert, angemeldete Gruppe 6 hat noch nicht reserviert}
  {Fehlermeldung, da Termin deaktiviert ist}
}

\newcommand{\testReserveBooked}{
\testCase{Teilsystem Termine anzeigen und buchen}
  {Termin reservieren}
  {Termin 11.7. aus Testdatensatz}
  {Termin ist bereits gebucht durch Gruppe 2, angemeldet als Gruppe 6}
  {Fehlermeldung, da Termin bereits gebucht ist}
}

\newcommand{\testReserve}{
\testCase{Teilsystem Termine anzeigen und buchen}
  {Termin reservieren}
  {Termin 20.7. aus Testdatensatz}
  {Als Gruppe 6 angemeldet, Termin ist frei}
  {Termin 20.7. ist von Gruppe 6 reserviert}
}

\newcommand{\testReserveGroupAlreadyBooked}{
\testCase{Teilsystem Termine anzeigen und buchen}
  {Termin reservieren}
  {Termin 13.7. aus Testdatensatz, Startzeit 11:00}
  {Angemeldete Gruppe 1 hat bereits gebucht}
  {Fehlermeldung, da Gruppe bereits gebucht hat}
}

\newcommand{\testCancelReservation}{
\testCase{Teilsystem Termine anzeigen und buchen}
  {Reservierung stornieren}
  {Termin 12.7. aus Testdatensatz}
  {Termin ist reserviert durch angemeldete Gruppe 2}
  {Termin ist wieder frei}
}

\newcommand{\testCancelOtherReservation}{
\testCase{Teilsystem Termine anzeigen und buchen}
  {Reservierung stornieren}
  {Termin 19.7. aus Testdatensatz}
  {Angemeldet als Gruppe 2, Termin gebucht durch Gruppe 4}
  {Fehlermeldung}
}

\newcommand{\testCancelReservationNoReservationPresent}{
\testCase{Teilsystem Termine anzeigen und buchen}
  {Reservierung stornieren}
  {Termin 23.7. aus Testdatensatz}
  {Termin ist nicht reserviert}
  {Fehlermeldung}
}

\newcommand{\testBook}{
\testCase{Teilsystem Termine anzeigen und buchen}
  {Termin buchen}
  {Termin 13.7. aus Testdatensatz, Startzeit 11:00}
  {Termin ist frei, angemeldet als Gruppe 5}
  {Termin ist durch Gruppe gebucht mit gewünschter Startzeit}
}

\newcommand{\testBookGroupBookedAlready}{
\testCase{Teilsystem Termine anzeigen und buchen}
  {Termin buchen}
  {Termin 24.7. aus Testdatensatz, Startzeit 11:00}
  {Angemeldete Gruppe 1 hat bereits gebucht}
  {Fehlermeldung, da Gruppe bereits gebucht hat}
}

\newcommand{\testBookGroupHasReservation}{
\testCase{Teilsystem Termine anzeigen und buchen}
  {Termin buchen}
  {Termin 25.7. aus Testdatensatz, Startzeit 11:00}
  {Angemeldet als Gruppe 4, die den 19.7. reserviert hat}
  {Termin ist durch Gruppe gebucht, Reservierung ist nicht mehr vorhanden}
}

\newcommand{\testBookInvalidStartTime}{
\testCase{Teilsystem Termine anzeigen und buchen}
  {Termin buchen}
  {Termin 26.7. aus Testdatensatz, Startzeit 11:00}
  {Startzeit ist außerhalb des Zeitfensters, angemeldete Gruppe  hat noch nicht gebucht}
  {Fehlermeldung, da Startzeit nicht in Zeitfenster}
}

\newcommand{\testBookReservedByOther}{
\testCase{Teilsystem Termine anzeigen und buchen}
  {Termin buchen}
  {Termin 12.7. aus Testdatensatz, Startzeit 11:00}
  {Angemeldet als Gruppe 7, Termin ist durch Gruppe 2 reserviert}
  {Fehlermeldung, da durch andere Gruppe reserviert}
}

%%%%%%%%%%%%%%%%%%%%% Ergebnisse %%%%%%%%%%%%%%%%%%%%%%%%%%%%%%%%%%%%%%%%%%%%%%%
\section{Testdokumentation}

\subsection{Unit Tests}
Die 4 Klassen des Pakets model wurden mittels JUnit getestet.

\subsection{Manuelle Systemtests}
\subsubsection{Testdaten}
Für die meisten Systemtests wurde der Testdatensatz der Tabellen \ref{tab:testAppointments} und \ref{tab:testGroups} verwendet.

\begin{table}
  \caption{Testdatensatz Termine}
  \label{tab:testAppointments}
  \begin{tabular}{|l|l|l|l|l|l|}
    \hline
    Datum & Zeitfenster & Zustand & Bemerkung & gebuchte/reservierte Gruppe & Buchungsdetails \\
    \hline \hline

    2018-07-09 & 07:30-16:40 & deaktiviert & NULL & & \\
    2018-07-10 & 07:30-16:40 & frei & NULL & & \\
    2018-07-11 & 07:30-16:40 & gebucht & NULL & Gruppe 1 & 11:00 \\
    2018-07-12 & 07:30-16:40 & reserviert & NULL & Gruppe 2 & \\
    2018-07-13 & 07:30-16:40 & frei & NULL & & \\
    \hline
    2018-07-16 & 07:30-16:40 & gebucht &NULL  & Gruppe 3 & 11:00 \\
    2018-07-17 & 07:30-16:40 & frei & NULL & & \\
    2018-07-18 & 07:30-16:40 & deaktiviert & NULL & & \\
    2018-07-19 & 07:30-16:40 & reserviert & NULL & Gruppe 4 & \\
    2018-07-20 & 07:30-16:40 & frei & NULL & & \\
    \hline
    2018-07-23 & 07:30-16:40 & frei & NULL & & \\
    2018-07-24 & 07:30-16:40 & frei & NULL & & \\
    2018-07-25 & 07:30-16:40 & frei & NULL & & \\
    2018-07-26 & 07:30-16:40 & frei & NULL & & \\
    2018-07-27 & 07:30-16:40 & frei & NULL & & \\
    \hline
  \end{tabular}
\end{table}

\begin{table}
  \caption{Testdatensatz Gruppen}
  \label{tab:testGroups}
  \begin{tabular}{|l|}
    \hline
    Gruppe \\
    \hline \hline
    1 \\
    2 \\
    3 \\
    4 \\
    6 \\
    7 \\
    \hline
  \end{tabular}
\end{table}

\testReport{7.7.2018}{Manfred Mustertester}{
  \testGenerate
}{15 Termine und 10 Gruppen}{Test uneingeschränkt bestanden}

\testReport{7.7.2018}{Manfred Mustertester}{
  \testReserveDeactivated
}{Termin wird reserviert}{Muss behoben werden, deaktivierter Termin darf nicht reserviert werden.}
