\documentclass[a4paper,10pt]{article}
\usepackage[utf8]{inputenc}
\usepackage{tocbibind}
\usepackage{xcolor}
\usepackage{tikz}
\usepackage{hyperref}
\usepackage[ngerman]{babel}
\usepackage{graphicx}
\graphicspath{ {./exports/png/} }


\begin{document}
\section{Grobentwurf}
Der Grobentwurf (siehe Abbildung \ref{fig:Paketdiagramm_Grobentwurf}) wurde nach der 3-Schichtenarchitektur aufgebaut. Genauer gesagt wurde innerhalb dieser das Model-View-Controller Muster verwendet. 
Dabei bilden die Views und die Controller die Präsentationsschicht und das Model die Logikschicht. Die Datenhaltungsschicht der 3-Schichtenarchitektur entspricht dem Paket Repository.

\subsection{Einzelne Elemente des MVC}
  \subsubsection{Models} 
    Die Models enthalten die Domänendaten. Sie werden durch ihre Methoden verändert. Die Models werden von allen Paketen Start, Student und Admin ben\"otigt. Dies geht aus der Abbildung \ref{fig:Paketdiagramm_Grobentwurf} hervor.
    Durch Ver\"anderungen in den Methoden reagieren die Models auf die Anferderungen.
   \subsubsection{Views}
   Die Views sind für die Darstellung der Daten und die Interaktion mit dem Nutzer zuständig, solbald der Nutzer zum Beispiel auf der Oberfläche einen Klick ausführt, wird dies sofort an den Controller weitergegeben, der wiederum das Model kontaktiert.
   Wie im Paketdiagramm in  Abbildung \ref{fig:Paketdiagramm_Grobentwurf} ersichtlich ist, enthalten die Pakete Start, Student, Admin jeweils einen oder mehrere Views.
   Au\ss{}erdem besitzen die Views eine Abh\"angigkeit zum Model.
   \subsubsection{Controller}
   Die Controller erhalten über die Models Zugriff zu den Daten und führen über diese Änderungen an den Daten aus. Dies kann man sehr gut in der Abbildung \ref{fig:Paketdiagramm_Grobentwurf} im Paketdiagramm sehen.
   
   \begin{figure}
    \includegraphics[scale=0.5]{Entwurf_PD_Grobentwurf.png}
    \label{fig:Paketdiagramm_Grobentwurf}
    \caption{Paketdiagramm Grobentwurf}
   \end{figure} 

  \newpage
\section{Feinentwurf}
Der Feinentwurf wurde zur besseren Übersicht in mehrere Klassendiagramme unterteilt.F\"ur die Pakete Start, Student, Admin, Model und Repository werden im Folgenden Klassendiagramme gezeigt.
Des Weiteren stellt das Klassendiagramm in Abbildung \ref{fig:Klassendiagramm_Models-Repo} das Repository und die Models dar. Es wird verwendet um Objekte auf Datenbankrelationen und Datenbankrelationen auf Objekte abzubilden. Da wir uns aus Gründen der Wartbarkeit für das MVC-Konzept entschieden haben,hat jeder Controller eine View.
Die Controller kommunizieren außerdem noch mit dem Model. Dieses besitzt als alleinige Komponente den Zugang zur Datenbank. Damit ist auch hier ohne großen Aufwand ein Austausch mit lokalen Daten möglich.
Durch diese Gestaltung der Komponenten haben wir außerdem eine bessere Wiederverwendbarkeit von Komponenten geschaffen. 

 \subsection{Die Klassendiagramme im Detail}
 \subsubsection{Das Paket Start}
 
 \begin{figure}
  \includegraphics[scale=0.5]{Entwurf_KD_Feinentwurf_Start.png}
  \label{fig:Klassendiagramm_Start}
  \caption{Feinentwurf des Pakets Start}
 \end{figure} 
 
 Das Klassendiagramm in Abbildung \ref{fig:Klassendiagramm_Start} zeigt die Klassen der Startansicht. Hier kann ausgewählt werden, welche Rolle der Benutzer einnimmt. 
 Daher gibt es hier nur eine View und einen Controller. Diese bilden die Rollenauswahl ab.
 
 \subsubsection{Das Paket Student}
 
 \begin{figure}
\includegraphics[scale=0.25]{Entwurf_KD_Feinentwurf_Student.png}
\label{fig:Klassendiagramm_Student}
\caption{Feinentwurf des Pakets Student}
 \end{figure} 

Das Klassendiagramm in Abbildung \ref{fig:Klassendiagramm_Student} zeigt die Klassen der Studentenansicht. Es ist in drei Controller und Views unterteilt. Eine zur Gruppenauswahl, für die Termine und eine für den Buchungsdialog.
 Jede View ist wieder mit dem entsprechenden Controller verbunden. Zusätzlich dazu gibt es Verbindungen zum Model um notwendige Daten der Termine und die Gruppen zu bestimmen.
 
 \subsubsection{Das Paket Admin}
 \begin{figure}
  \includegraphics[scale=0.25]{Entwurf_KD_Feinentwurf_Admin.png}
  \label{fig:Klassendiagramm_Admin}
  \caption{Feinentwurf des Pakets Admin}
 \end{figure} 

 Die Abbildung \ref{fig:Klassendiagramm_Admin} zeigt die Klassen der Adminansicht. Es gibt vier Controller mit den dazugehörigen Views. Der Controller für die Appointments und für den Bearbeitungsdialog benötigen im Model die Appointments. Der Groupcontroller benötigt Zugriff auf die Gruppen im Model.
 Der SetupController ist f\"ur die Erstellung der Gruppen, Verbindung zur Datenbank und dem erstellen des leeren Pr\"ufungsplans verantwortlich.Der AppointmentController verwaltet die Termine, der EditDialogController ist f\"ur die genen Zeiten des Termines oder f\"ur Eintragungen des Raumes notwednig.
 Der GroupController ist f\"ur die Verwaltung der Gruppen zust\"andig.
 \subsubsection{Model und Repository}
 
 \begin{figure}
  \includegraphics[scale=0.20]{Entwurf_KD_Models-Repo.png}
  \label{fig:Klassendiagramm_Models-Repo}
\caption{Feinentwurf von Model und das Repository}
 \end{figure} 

Im Paket Model (siehe Abbildung \ref{fig:Klassendiagramm_Models-Repo}) befinden sich die Klassen Appointment, Reservation, Booking, TimeWindow und Group. Die Klasse Appointment ist für die Termine zuständig. Die Klasse Reservation kümmert sich um Reservierungen. Booking um das Buchen von Terminen, TimeWindow um die Zeitfenster der Termine und Buchungen und die Klasse Group um alle Belange der Gruppen, wie zum Beispiel anlegen oder löschen.
Im Paket Repository (ebenfalls in Abbildung \ref{fig:Klassendiagramm_Models-Repo}) befindet sich die Klasse DBRepository. Sie kümmert sich um die Verbindung zur Datenbank und gibt zum Beipiel alle Gruppen aus oder fügt Termine ein.
In der Datei db.properties werden die Zugangsdaten abgelegt. Dadurch ist es m\"oglich eine andere Datenbank zu w\"ahlen.
Wichtig zu erw\"ahnen sind die Klassen Reservation, Booking und TimeWindow. Ein Appointment besteht unter anderem aus einem TimeWindow, daher gibt es eine Kompositionsbeziehung zwischen diesen beiden Klassen.
\end{document}
