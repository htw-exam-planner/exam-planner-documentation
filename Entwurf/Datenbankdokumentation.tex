\documentclass[10pt,a4paper]{article}
\usepackage[utf8]{inputenc}
\usepackage[german]{babel}
\usepackage{graphicx}
\usepackage[table,x11names]{xcolor}


\begin{document}

Die Datenbank für die Applikation besteht aus vier Tabellen : Group, Reservation, Appointment und Booking.
Die Struktur der Tabellen ist in Abbildung \ref{fig:ER-Diagramm} dargestellt.

\begin{figure}[ht]		
	\centering
	\includegraphics[scale=0.3]{ER-Diagramm}
	\caption{Datenbankentwurf}
	\label{fig:ER-Diagramm}	
	\end{figure}

Abkürzungserläuterung
\begin{itemize}
\item N : NULLABALE
\item P : PRIMARRYKEY
\item F : FOREINGKEY
\end{itemize}

\begin{table}[h]
\centering
\caption{Appointment}
	\label{tab:Appointment}   
    \begin{tabular}{| l | l | l | l |}
    \hline
    \rowcolor{lightgray} Spalte & Datentyp & Beschreibung  \\ \hline
    Date  & DATE & Datum des Termins(Primarykey).\\ \hline
    Activated & BIT & Ob der Termin an den gewünschten Tag deaktiviert/aktiviert wird.\\ \hline
    Startime & TIME & Beginnzeit des Zeitfensters. \\ \hline
    Endtime & TIME & Endezeit des Zeitfensters.  \\ \hline
    Note & VARCHAR(128) & Notizen für Appointment(Nullable). \\ \hline
    \end{tabular}
\end{table}

In Tabelle \ref{tab:Appointment} wird die Struktur der Tabelle Appointment dargestellt. Die Tabelle Appointment speichert alle Termine. Mit dem Feld "Activated" wird gekennzeichnet, ob ein Termin deaktiviert oder aktiviert ist. Die Zustände "gebucht" und "reserviert" ergeben sich daraus, dass ein entsprechender Eintrag in "Booking" oder "Reservation" vorhanden ist.
\begin{itemize}
\item Es darf nur eine Presentation pro tag geben.
\item Jeder Termin darf nur von einem Eintrag in der Tabelle Reservation referenziert werden. 
\end{itemize}

\begin{table}[h]
\centering
	\caption{Reservation}
	\label{tab:Reservation}
    \begin{tabular}{| l | l | l |}
    \hline
    \rowcolor{lightgray} Spalte & Datentyp & Beschreibung  \\ \hline
    Groups & INTEGER & Die Gruppe, die den Termin reserviert oder gebucht hat(Primarykey, Foreign key auf Groups.Groupnumber).  \\ \hline
    Appoinment & DATE & Der Termin, der durch die Gruppe gebucht oder reserviert ist(Foreign key auf Appointment.Date). \\ \hline
    \end{tabular}
\end{table}
In Tabelle \ref{tab:Reservation} wird die Struktur der Tabelle Reservation  dargestellt.  In dieser Tabelle werden sowohl Reservierungen, als auch Buchungen eingetragen. Eine Buchung ist dadurch zu erkennen, dass es zusätzlich einen entsprechenden Eintrag in Booking gibt. Der Grund für diese Art der Darstellung ist die Erleichterung der Integritätsprüfung dass jeder Termin max. eine Buchung oder Reservierung (egal welches) haben darf, und auch jede Gruppe max. 1 Buchung oder Reservierung.

\begin{itemize} 
\item Jeder Termin kann nur einen Eintrag in Reservierung haben.
\item Jede Gruppe darf nur einen Eintrag in Reservierung haben.
\item Einen Termin kann mit einer Reservierung oder gar keine ausgemacht werden.
\end{itemize}
\begin{table}[h]
\centering
\caption{Groups}
    \label{tab:Groups}    
    \begin{tabular}{| l | l | l | l |}
    \hline
    \rowcolor{lightgray} Spalte & Datentyp & Beschreibung  \\ \hline
    Groupnumber & INTEGER & Groupsnummer(Primarykey).  \\ \hline
   \end{tabular}
\end{table}

\begin{itemize}
\item Jede Gruppe kann entweder eine Reservierung oder gar keine haben. 

\end{itemize}

\begin{table}[h]
\centering
\caption{Booking}
	\label{tab:Booking}
    \begin{tabular}{| l | l | l | l |}
    \hline
    \rowcolor{lightgray} Spalte & Datentyp & Beschreibung  \\ \hline
    Reservation  & INTEGER & Referenz auf den dazugehörigen Eintrag in Reservation(Primarykey,Foreign key Reservation.Groups). \\ \hline
    Startime & TIME & Startzeit der Buchung. \\ \hline
    Endtime & TIME & Endzeit der Buchung. \\ \hline
    Room & CHAR(15) & Raum wo der Termin stattfindet. \\ \hline
    \end{tabular}
\end{table}

In Tabelle \ref{tab:Booking} wird die Struktur der Tabelle Booking dargestellt. Die Tabelle Booking dient als ein Speziallsfall von der Tabele Reservation. Weil jede Gruppe max. 1 Reservierung oder Buchung haben kann.

\end{document}