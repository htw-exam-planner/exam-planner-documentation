\documentclass[10pt,a4paper]{article}
\usepackage[utf8]{inputenc}
\usepackage[german]{babel}
\usepackage[T1]{fontenc}
\usepackage{amsmath}
\usepackage{amsfonts}
\usepackage{amssymb}
\usepackage{graphicx}
\usepackage[table,x11names]{xcolor}
\newcommand\sbullet[1][.5]{\mathbin{\vcenter{\hbox{\scalebox{#1}{$\bullet$}}}}}
\graphicspath{ {/home/ducio/Desktop/Se/exam-planner-documentation/Entwurf} }

\begin{document}

Datenbank für das Applikation besteht aus  vier Tabele : Group, Reservation, Appointment und Booking.

\begin{table}[h]
\centering
	\caption{Reservation}
    \begin{tabular}{| l | l | l |}
    \hline
    \rowcolor{lightgray} Columm & Datentyp & Beschreibung  \\ \hline
    Groups & INTEGER & Gruppesnummer(Primarykey, Foreign key auf Groups.Groupnumber).  \\ \hline
    Appoinment & DATE & Terminsvreservierung(Foreign key auf Appointment.Date). \\ \hline
    \end{tabular}
\end{table}
Logikserläuterung zur Tabelle Rerservation

\begin{itemize}
\item Gruppesnummer ist eindeutig, 
\item Terminsreservierung kann nur einen Termin( Appointment) haben.
\item Terminsreservierung darf nur von einer Gruppe ausgemacht werden.
\end{itemize}
\begin{table}[h]
\centering
\caption{Groups}
    \begin{tabular}{| l | l | l | l |}
    \hline
    \rowcolor{lightgray} Columm & Datentyp & Beschreibung  \\ \hline
    Groupnumber & INTEGER & Groupsnummer(Primarykey).  \\ \hline
   \end{tabular}
\end{table}
Logiksserläuterung zur Tabelle Groups
\begin{itemize}
\item Jede Gruppe kann entweder eine Reservierung oder gar keine haben. 
\item Gruppesnummer ist eindeutig.
\end{itemize}

\begin{table}[h]
\centering
\caption{Booking}
    \begin{tabular}{| l | l | l | l |}
    \hline
    \rowcolor{lightgray} Columm & Datentyp & Beschreibung  \\ \hline
    Reservation  & INTEGER & Terminsbuchung(Primarykey,Foreign key Reservation.Groups). \\ \hline
    Startime & TIME & Startzeit des Termins. \\ \hline
    Endtime & TIME & Endzeit des Termins. \\ \hline
    Room & CHAR(15) & Raum wo der Termin stattfindet. \\ \hline
    \end{tabular}
\end{table}


Logikserläuterung zur Tabelle Booking
\begin{itemize}
\item ich weiß nicht mehr wie ich beschreiben soll?
\end{itemize}

\begin{table}[h]
\centering
\caption{Appointment}
    \begin{tabular}{| l | l | l | l |}
    \hline
    \rowcolor{lightgray} Columm & Datentyp & Beschreibung  \\ \hline
    Date  & DATE & Datum des Gruppepresentations(Primarykey).\\ \hline
    Activated & BIT & Zustand des Termins, ob der Termin an den gewünschten Tag freigeschaltet wird.\\ \hline
    Startime & TIME & Beginnzeit des Grouppresentations. \\ \hline
    Endtime & TIME & Endezeit des Grouppresentations.  \\ \hline
    Note & VARCHAR(128) & Notizen für Appointment(Nullable). \\ \hline
    \end{tabular}
\end{table}

Logikserläterung zur Tabelle Appointment
\begin{itemize}
\item Datum des Gruppepresentations ist eindeutig,es darf nur eine Presentation pro tag geben.
\item Einen Termin kann mit eine Reservierung oder gar keine ausgemacht werden, 
\end{itemize}

\begin{figure}[ht]
	\centering
	\includegraphics[scale=0.3]{ER-Diagramm}
	\caption{Datenbankentwurf}	
	\end{figure}

\end{document}




