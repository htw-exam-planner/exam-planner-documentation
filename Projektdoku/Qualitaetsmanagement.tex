\section{Qualitätsmanagement}
Um die Qualität des Projekergebnisses und -ablaufs sicherzustellen wurden die folgenden Qualitätsrichtlinien festgelegt.

\begin{itemize}
\item Der gesamte Quellcode einschließlich Kommentaren wird in Englisch verfasst.
\item Alle öffentlichen Methoden werden mit Javadoc dokumentiert.
\item Die erste logische Ebene des Quellcodes wird mit einem Tab und jede weitere logische Ebenen mit einem Tab mehr als die Vorgängerebene eingerückt.
\item Eingaben werden immer überprüft und mögliche Fehler durch das Werfen von Exceptions behandelt.
\item Der Quellcode folgt dem Prinzip KISS (Keep it stupid simple) und DRY (Don't repeat yourself).
\item Die maximale Länge von 150 Zeichen Quellcode pro Zeile darf nicht überschritten werden.
\item Die Implementierung muss jederzeit durch andere Projektmitglieder verstanden werden können.
\item Jegliche Änderungen am Quellcode werden in einem extra Branch verwaltet und nur über einen Pull-Request in den master-Branch integriert. Jeder Pull-Request wird von 2 Personen, die nicht an der Implemntierung dieses Pull-Requests beteiligt waren, überprüft.
\item Das Softwaresystem besitzt mehrere Klassen, die nach Modulen unterteilt sind.
\item Der Quellcode des Softwaresystems wird versioniert und auf Github allen Mitgliedern zugänglich gemacht.
\end{itemize}
