%%%%%%%%%%%%%%%%%%%%%%%%% Makros %%%%%%%%%%%%%%%%%%%%%%%%%%%%%%%%%%%%%%%%%%%%%%%

%Semanktik: \testReport{Datum}{Tester}{Testfall}{Ergebnis}{Bewertung}
\newcommand{\testReport}[5]{
  \paragraph{Test am #1, durchgeführt von #2}
  #3
  \subparagraph{Tatsächliches Ergebnis:} #4
  \subparagraph{Bewertung:} #5
}

%Semanktik: \testCase{Gegenstand}{Testfall}{Testdaten}{Rahmenbedingungen}{Erwartetes Ergebnis}
\newcommand{\testCase}[5]{
  \subparagraph{Testgegenstand:} #1
  \subparagraph{Testfall:} #2
  \subparagraph{Testdaten:} #3
  \subparagraph{Rahmenbedingungen:} #4
  \subparagraph{Erwartetes Ergebnis:} #5
}

%%%%%%%%%%%%%% Testfälle %%%%%%%%%%%%%%%%%%%%%%%%%%%%%%%%%%%%%%%%%%%%%%%%%%%%%%%
\newcommand{\testGenerate}{
  \testCase{Teilsystem Prüfungstermine verwalten}
    {Termine und Gruppen generieren}
    {Startdatum: 9.7.2018, Anzahl Gruppen: 10}
    {}
    {15 freie Termine an den Wochentagen ab 9.7., 10 Gruppen}
}

\newcommand{\testGenerateNotMonday}{
  \testCase{Teilsystem Prüfungstermine verwalten}
    {Termine und Gruppen generieren}
    {Startdatum: 10.7.2018, Anzahl Gruppen: 10}
    {}
    {Warnhinweis, da 10.7. kein Montag ist}
}

\newcommand{\testGenerateZeroGroups}{
  \testCase{Teilsystem Prüfungstermine verwalten}
    {Termine und Gruppen generieren}
    {Startdatum: 9.7.2018, Anzahl Gruppen: 0}
    {}
    {Warnhinweis, da 0 Gruppen unzulässig sind}
}

\newcommand{\testFreeDeactivated}{
  \testCase{Teilsystem Prüfungstermine verwalten}
    {Termin als frei markieren}
    {deaktivierter Termin 9.7.2018 aus Testdatensatz}
    {Termin ist deaktiviert}
    {Termin 9.7. hat Zustand frei, ansonsten keine Änderung}
}

\newcommand{\testFreeBooked}{
  \testCase{Teilsystem Prüfungstermine verwalten}
    {Termin als frei markieren}
    {Termin 11.7.2018, gebucht von Gruppe 1 aus Testdatensatz}
    {Termin ist gebucht}
    {Termin 11.7. hat Zustand frei, die Buchung ist nicht mehr vorhanden}
}

\newcommand{\testDeactivateFree}{
  \testCase{Teilsystem Prüfungstermine verwalten}
    {Termin deaktivieren}
    {Termin 10.7. aus Testdatensatz}
    {Termin ist frei}
    {Termin ist deaktiviert, ansonsten keine Änderung}
}

\newcommand{\testDeactivateReserved}{
\testCase{Teilsystem Prüfungstermine verwalten}
  {Termin deaktivieren}
  {Termin 12. 7. aus Testdatensatz}
  {Termin ist reserviert}
  {Termin ist deaktiviert, die Reservierung ist nicht mehr vorhanden}
}

\newcommand{\testSetTimeWindow}{
\testCase{Teilsystem Prüfungstermine verwalten}
  {Zeitfenster bearbeiten}
  {Termin 13.7. aus Testdatensatz, Zeitfenster 08:00-12:00}
  {}
  {Termin mit neuem Zeitfenster 08:00-12:00}
}

\newcommand{\testSetTimeWindowInvalid}{
\testCase{Teilsystem Prüfungstermine verwalten}
  {Zeitfenster bearbeiten}
  {Termin 13.7. aus Testdatensatz, Zeitfenster 12:00-08:00}
  {}
  {Fehlermeldung}
}

\newcommand{\testSetNote}{
\testCase{Teilsystem Prüfungstermine verwalten}
  {Bemerkung bearbeiten}
  {Termin 13.7. aus Testdatensatz, Bemerkung "`Hier könnte Ihre Werbung stehen"'}
  {}
  {Termin hat neue Bemerkung "`Hier könnte Ihre Werbung stehen"'}
}

\newcommand{\testSetBookingRoomTime}{
\testCase{Teilsystem Prüfungstermine verwalten}
  {Raum und Endzeit einer Buchung festlegen}
  {Termin 16.7. aus Testdatensatz, Endzeit 15:00, Raum Z146b}
  {Termin ist durch Gruppe 3 gebucht}
  {Termin mit Endzeit und Raum wie festgelegt}
}

\newcommand{\testSetBookingRoomTimeNotBooked}{
\testCase{Teilsystem Prüfungstermine verwalten}
  {Raum und Endzeit einer Buchung festlegen}
  {Termin 13.7. aus Testdatensatz}
  {Termin ist nicht gebucht}
  {Fehlermeldung, da Termin nicht gebucht}
}

\newcommand{\testShowAppointments}{
\testCase{Teilsystem Termine anzeigen}
  {Termine anzeigen}
  {}
  {Termine wurden bereits generiert}
  {Übersicht der Termine}
}

\newcommand{\testViewGroups}{
\testCase{Teilsystem Gruppen verwalten}
  {Gruppen anzeigen}
  {}
  {Gruppen wurden bereits generiert}
  {Übersicht der Gruppen}
}

\newcommand{\testDeleteGroup}{
\testCase{Teilsystem Gruppen verwalten}
  {Gruppe löschen}
  {Gruppe 3}
  {Gruppe 3 hat Termin 16.7. gebucht}
  {Gruppe ist nicht mehr vorhanden, Buchung am 16.7. nicht mehr vorhanden}
}

\newcommand{\testCreateGroup}{
\testCase{Teilsystem Gruppen verwalten}
  {Gruppe anlegen}
  {}
  {Gruppen des Testdatensatzes vorhanden, Gruppe 3 ist gelöscht}
  {Neue Gruppe mit Nummer 8 angelegt}
}

\newcommand{\testLogInGroup}{
\testCase{Teilsystem Termine anzeigen und buchen}
  {Als Gruppe anmelden}
  {Gruppe 6}
  {}
  {Terminübersicht}
}

\newcommand{\testReserveDeactivated}{
\testCase{Teilsystem Termine anzeigen und buchen}
  {Termin reservieren}
  {Termin 9.7. aus Testdatensatz}
  {Termin ist deaktiviert, angemeldete Gruppe 6 hat noch nicht reserviert}
  {Fehlermeldung, da Termin deaktiviert ist}
}

\newcommand{\testReserveBooked}{
\testCase{Teilsystem Termine anzeigen und buchen}
  {Termin reservieren}
  {Termin 11.7. aus Testdatensatz}
  {Termin ist bereits gebucht durch Gruppe 2, angemeldet als Gruppe 6}
  {Fehlermeldung, da Termin bereits gebucht ist}
}

\newcommand{\testReserve}{
\testCase{Teilsystem Termine anzeigen und buchen}
  {Termin reservieren}
  {Termin 20.7. aus Testdatensatz}
  {Als Gruppe 6 angemeldet, Termin ist frei}
  {Termin 20.7. ist von Gruppe 6 reserviert}
}

\newcommand{\testReserveGroupAlreadyBooked}{
\testCase{Teilsystem Termine anzeigen und buchen}
  {Termin reservieren}
  {Termin 13.7. aus Testdatensatz, Startzeit 11:00}
  {Angemeldete Gruppe 1 hat bereits gebucht}
  {Fehlermeldung, da Gruppe bereits gebucht hat}
}

\newcommand{\testCancelReservation}{
\testCase{Teilsystem Termine anzeigen und buchen}
  {Reservierung stornieren}
  {Termin 12.7. aus Testdatensatz}
  {Termin ist reserviert durch angemeldete Gruppe 2}
  {Termin ist wieder frei}
}

\newcommand{\testCancelOtherReservation}{
\testCase{Teilsystem Termine anzeigen und buchen}
  {Reservierung stornieren}
  {Termin 19.7. aus Testdatensatz}
  {Angemeldet als Gruppe 2, Termin gebucht durch Gruppe 4}
  {Fehlermeldung}
}

\newcommand{\testCancelReservationNoReservationPresent}{
\testCase{Teilsystem Termine anzeigen und buchen}
  {Reservierung stornieren}
  {Termin 23.7. aus Testdatensatz}
  {Termin ist nicht reserviert}
  {Fehlermeldung}
}

\newcommand{\testBook}{
\testCase{Teilsystem Termine anzeigen und buchen}
  {Termin buchen}
  {Termin 13.7. aus Testdatensatz, Startzeit 09:00}
  {Termin ist frei, angemeldet als Gruppe 5}
  {Termin ist durch Gruppe gebucht mit gewünschter Startzeit}
}

\newcommand{\testBookGroupBookedAlready}{
\testCase{Teilsystem Termine anzeigen und buchen}
  {Termin buchen}
  {Termin 24.7. aus Testdatensatz, Startzeit 09:00}
  {Angemeldete Gruppe 1 hat bereits gebucht}
  {Fehlermeldung, da Gruppe bereits gebucht hat}
}

\newcommand{\testBookGroupHasReservation}{
\testCase{Teilsystem Termine anzeigen und buchen}
  {Termin buchen}
  {Termin 25.7. aus Testdatensatz, Startzeit 09:00}
  {Angemeldet als Gruppe 4, die den 19.7. reserviert hat}
  {Termin ist durch Gruppe gebucht, Reservierung ist nicht mehr vorhanden}
}

\newcommand{\testBookInvalidStartTime}{
\testCase{Teilsystem Termine anzeigen und buchen}
  {Termin buchen}
  {Termin 26.7. aus Testdatensatz, Startzeit 07:00}
  {Startzeit ist außerhalb des Zeitfensters, angemeldete Gruppe 6 hat noch nicht gebucht}
  {Fehlermeldung, da Startzeit nicht in Zeitfenster}
}

\newcommand{\testBookReservedByOther}{
\testCase{Teilsystem Termine anzeigen und buchen}
  {Termin buchen}
  {Termin 12.7. aus Testdatensatz, Startzeit 09:00}
  {Angemeldet als Gruppe 7, Termin ist durch Gruppe 2 reserviert}
  {Fehlermeldung, da durch andere Gruppe reserviert}
}

\newcommand{\testCreateGroupAndBook}{
\testCase{Teilsysteme Termine anzeigen und buchen und Termine verwalten}
  {Gruppe anlegen und Termin buchen}
  {Termin 26.7. aus Testdatensatz, Startzeit 09:00}
  {Angemeldet als neu angelegte Gruppe}
  {Termin ist gebucht}
}

%%%%%%%%%%%%%%%%%%%%% Ergebnisse %%%%%%%%%%%%%%%%%%%%%%%%%%%%%%%%%%%%%%%%%%%%%%%
\section{Testdokumentation}

\subsection{Unit Tests}
Ziel war es, das Funktionieren der einzelnen Teilkomponenten mittels Unit Tests sicherzustellen. Dafür wurde eine Infrastruktur für Unit Tests konfiguriert und Unit Tests geschrieben. Aufgrund eines unvorteilhaften Systementwurfes ergaben sich allerdings Schwierigkeiten, die das Unit Testen der meisten wichtigen Komponenten unmöglich machte. Aufgrund des kurz bevorstehenden Projektendes war es nicht mehr möglich den Entwurf entsprechend anzupassen. Daraufhin wurde sich dafür entschieden auf weitere Unit Tests zu verzichten und die Funktionalität der Software ausschließlich mithilfe der manuellen Systemtests sicherzustellen. Aufgrund der kleinen Systemgröße schien dieses Vorgehen zwar unvorteilhaft, aber in Anbetracht der zeitlichen Umstände als ausreichend.

\subsubsection{Unit Test Infrastruktur}
\paragraph{Test Formulierung}
Die Unit Tests wurden mittels dem Test Framework JUnit 4 \href{https://junit.org}{https://junit.org} realisiert. Dazu wurden für die zu testenden Klassen entsprechende Testklassen im Verzeichnis \texttt{src/test/java} angelegt. Test-Daten-Objekte (Mocks) wurden mittels Mockito \href{http://site.mockito.org}{http://site.mockito.org} erzeugt. Für das Abprüfen der Testergebnisse wurde AssertJ \href{http://joel-costigliola.github.io/assertj/}{http://joel-costigliola.github.io/assertj/} verwendet.
\paragraph{Test Ausführung}
Die Unit Tests werden über die standardmäßig verfügbare Testausführung von Maven ausgeführt. Diese führt alle Tests in \texttt{src/test/java} aus. Dazu muss das Kommando \texttt{mvn verify} ausgeführt werden.

\subsubsection{Probleme beim Schreiben von Unit Tests}
Der Zugriff der Anwendungskomponenten auf die Model-Daten wurde über statische Methoden auf den Model-Klassen realisiert, die den Zugriff auf die Datenhaltungsschicht durchführen. Eine technische Einschränkung der Java Programmiersprache ist, dass es nicht ohne sehr großen Aufwand möglich ist, Aufrufe auf statische Methoden zu Testzwecken durch einen "gefälschten" (gemockten) Aufruf zu ersetzen. Da die meisten Presenter und Models von statischen Methoden anderer Models abhingen, war es somit nicht möglich diese Komponenten isoliert zu testen. Des Weiteren wurde zum Zwecke der Vereinfachung auch auf Entkopplungsmechanismen, wie Dependency Injection, verzichtet, sodass eine hohe Kopplung zwischen den Komponenten und ihren Abhängigkeiten existiert. Dies führt dazu, dass sich diese Abhängigkeiten nicht zu Testzwecken gegen Test-Daten-Objekte austauschen ließen. Dadurch konnten bestimmte Klassen (z.B. \texttt{DBRepository}) ebenfalls nicht getestet werden.

\subsection{Manuelle Systemtests}
\subsubsection{Testdaten}
Für die meisten Systemtests wurde der Testdatensatz der Tabellen \ref{tab:testAppointments} und \ref{tab:testGroups} verwendet.

\begin{table}
  \caption{Testdatensatz Termine}
  \label{tab:testAppointments}
  \begin{tabular}{|l|l|l|l|l|l|}
    \hline
    Datum & Zeitfenster & Zustand & Bemerkung & gebuchte/reservierte Gruppe & Buchungsdetails \\
    \hline \hline

    2018-07-09 & 07:30-16:40 & deaktiviert & NULL & & \\
    2018-07-10 & 07:30-16:40 & frei & NULL & & \\
    2018-07-11 & 07:30-16:40 & gebucht & NULL & Gruppe 1 & 09:00 \\
    2018-07-12 & 07:30-16:40 & reserviert & NULL & Gruppe 2 & \\
    2018-07-13 & 07:30-16:40 & frei & NULL & & \\
    \hline
    2018-07-16 & 07:30-16:40 & gebucht &NULL  & Gruppe 3 & 09:00 \\
    2018-07-17 & 07:30-16:40 & frei & NULL & & \\
    2018-07-18 & 07:30-16:40 & deaktiviert & NULL & & \\
    2018-07-19 & 07:30-16:40 & reserviert & NULL & Gruppe 4 & \\
    2018-07-20 & 07:30-16:40 & frei & NULL & & \\
    \hline
    2018-07-23 & 07:30-16:40 & frei & NULL & & \\
    2018-07-24 & 07:30-16:40 & frei & NULL & & \\
    2018-07-25 & 07:30-16:40 & frei & NULL & & \\
    2018-07-26 & 07:30-16:40 & frei & NULL & & \\
    2018-07-27 & 07:30-16:40 & frei & NULL & & \\
    \hline
  \end{tabular}
\end{table}

\begin{table}
  \caption{Testdatensatz Gruppen}
  \label{tab:testGroups}
  \begin{tabular}{|l|}
    \hline
    Gruppe \\
    \hline \hline
    1 \\
    2 \\
    3 \\
    4 \\
    6 \\
    7 \\
    \hline
  \end{tabular}
\end{table}

\subsubsection{Tests des Teilsystems Termine anzeigen und Buchen am 13.7.}
\testReport{13.7.2018 13:00}{Oliver von Seydlitz}{
  \testLogInGroup
}{Terminübersicht}{Test bestanden}

\testReport{13.7.2018 13:00}{Oliver von Seydlitz}{
  \testReserveDeactivated
}{Oberfläche lässt Reservierung eines deaktivieren Termins nicht zu}{Test bestanden}

\testReport{13.7.2018 13:00}{Oliver von Seydlitz}{
  \testReserveBooked
}{Oberfläche lässt Reservierung eines gebuchten Termins nicht zu}{Test bestanden}

\testReport{13.7.2018 13:00}{Oliver von Seydlitz}{
  \testReserve
}{Termin 20.7. ist von Gruppe reserviert, Reservierungstext ist nicht komplett sichtbar}{Test bestanden, jedoch mit Makel. Das Textfeld für den Zustand sollte verbreitert werden}

\testReport{13.7.2018 13:00}{Oliver von Seydlitz}{
  \testReserveGroupAlreadyBooked
}{Oberfläche lässt Reservierung nicht zu, wenn bereits gebucht ist}
{Test bestanden}

\testReport{13.7.2018 13:00}{Oliver von Seydlitz}{
  \testCancelReservation
}{Termin ist frei, Gruppe 2 hat keine Reservierung mehr}
{Test bestanden}

\testReport{13.7.2018 13:00}{Oliver von Seydlitz}{
  \testCancelOtherReservation
}{Oberfläche lässt Stornierung nicht zu, wenn der Termin durch andere Gruppe reserviert ist}
{Test bestanden}

\testReport{13.7.2018 13:00}{Oliver von Seydlitz}{
  \testCancelReservationNoReservationPresent
}{Oberfläche lässt Stornierung nicht zu, wenn der nicht reserviert ist}
{Test bestanden}

\testReport{13.7.2018 13:00}{Oliver von Seydlitz}{
  \testBook
}{Nichts passiert}
{Test nicht bestanden. Ursache: OperationNotAllowedException, da Zeitfenster falsch in der Datenbank eingetragen werden. Maßnahme: Entsprechende Stelle in der Klasse DBRepository korrigieren}

Nach Behebung dieses Fehlers wurden dann weitere Tests zum Anwendungsfall "`Termin buchen"' durchgeführt.

\testReport{13.7.2018 13:30}{Oliver von Seydlitz}{
  \testBook
}{Termin ist durch Gruppe 5 gebucht}
{Test bestanden}

\testReport{13.7.2018 13:30}{Oliver von Seydlitz}{
  \testBookGroupBookedAlready
}{Oberfläche lässt Buchung nicht zu, wenn Gruppe bereits gebucht hat}
{Test bestanden}

\testReport{13.7.2018 13:30}{Oliver von Seydlitz}{
  \testBookGroupHasReservation
}{Oberfläche lässt Buchung eines anderen Termins nicht zu, wenn die Gruppe bereits reserviert hat}
{Defekt. Die Buchung anderer Termine sollte möglich sein, ohne die Reservierung stornieren zu müssen. Die Oberfläche sollte angepasst werden.}

\testReport{13.7.2018 13:30}{Oliver von Seydlitz}{
  \testBookReservedByOther
}{Oberfläche lässt Buchung eines durch eine andere Gruppe reservierten Termins nicht zu}
{Test bestanden}

\testReport{13.7.2018 18:00}{Oliver von Seydlitz}{
  \testReserve
}{Termin 20.7. ist von Gruppe reserviert, Reservierungstext nun vollkommen sichtbar}{Test bestanden}

\testReport{13.7.2018 18:00}{Oliver von Seydlitz}{
  \testBookGroupHasReservation
}{Termin 25.7. ist durch Gruppe 4 gebucht, die Reservierung ist nicht mehr vorhanden}
{Test bestanden}

\subsubsection{Systemtests am 15. 7. 2018}

\testReport{15.7.2018 16:20}{Oliver von Seydlitz}{
  \testGenerate
}{15 freie Termine mit Zeitfenster 8:30-16:40, 10 Gruppen}
{Defekt: Startzeit sollte 7:30 sein, sollte behoben werden}

\testReport{15.7.2018 16:20}{Oliver von Seydlitz}{
  \testGenerateNotMonday
}{Fehlermeldung}
{Test bestanden}

\testReport{15.7.2018 16:20}{Oliver von Seydlitz}{
  \testGenerateZeroGroups
}{Fehlermeldung}
{Test bestanden}

\testReport{15.7.2018 16:20}{Oliver von Seydlitz}{
  \testFreeDeactivated
}{Termin 9.7. ist frei}
{Test bestanden}

\testReport{15.7.2018 16:20}{Oliver von Seydlitz}{
  \testFreeBooked
}{Termin 9.7. hat Zustand frei, die Buchung ist nicht mehr vorhanden}
{Test bestanden}

\testReport{15.7.2018 16:20}{Oliver von Seydlitz}{
  \testDeactivateFree
}{Termin 10.7. ist deaktiviert}
{Test bestanden}

\testReport{15.7.2018 16:20}{Oliver von Seydlitz}{
  \testDeactivateReserved
}{Termin ist deaktiviert, die Reservierung ist nicht mehr vorhanden}
{Test bestanden}

\testReport{15.7.2018 16:20}{Oliver von Seydlitz}{
  \testSetTimeWindow
}{Termin mit neuem Zeitfenster 08:00-12:00}
{Test bestanden}

\testReport{15.7.2018 16:20}{Oliver von Seydlitz}{
  \testSetTimeWindowInvalid
}{Fehlermeldung}
{Test bestanden}

\testReport{15.7.2018 16:30}{Oliver von Seydlitz}{
  \testSetNote
}{Termin hat neue Bemerkung "`Hier könnte Ihre Werbung stehen"'}
{Test bestanden}

\testReport{15.7.2018 16:30}{Oliver von Seydlitz}{
  \testSetBookingRoomTime
}{Termin mit Endzeit und Raum wie festgelegt}
{Test bestanden}

\testReport{15.7.2018 16:30}{Oliver von Seydlitz}{
  \testSetBookingRoomTimeNotBooked
}{Oberläche lässt Bearbeitung der Buchung nur zu, wenn Buchung vorhanden}
{Test bestanden}

\testReport{15.7.2018 16:30}{Oliver von Seydlitz}{
  \testShowAppointments
}{Übersicht der Termine}
{Test bestanden}

\testReport{15.7.2018 16:30}{Oliver von Seydlitz}{
  \testViewGroups
}{Übersicht der Gruppen}
{Test bestanden}

\testReport{15.7.2018 16:30}{Oliver von Seydlitz}{
  \testCreateGroup
}{Neue Gruppe mit Nummer 8 angelegt}
{Test bestanden}

\testReport{15.7.2018 16:30}{Oliver von Seydlitz}{
  \testDeleteGroup
}{Gruppe ist nicht mehr vorhanden, Buchung am 16.7. nicht mehr vorhanden}
{Test bestanden}

\testReport{15.7.2018 16:35}{Oliver von Seydlitz}{
  \testLogInGroup
}{Terminübersicht}
{Test bestanden}

\testReport{15.7.2018 16:35}{Oliver von Seydlitz}{
  \testReserveDeactivated
}{Oberfläche lässt Reservierung eines deaktivieren Termins nicht zu}
{Test bestanden}

\testReport{15.7.2018 16:35}{Oliver von Seydlitz}{
  \testReserveBooked
}{Oberfläche lässt Reservierung eines gebuchten Termins nicht zu}
{Test bestanden}

\testReport{15.7.2018 16:40}{Oliver von Seydlitz}{
  \testReserve
}{Termin 20.7. ist von Gruppe 6 reserviert}
{Test bestanden}

\testReport{15.7.2018 16:40}{Oliver von Seydlitz}{
  \testReserveGroupAlreadyBooked
}{Oberfläche lässt Reservierung nicht zu, wenn bereits gebucht ist}
{Test bestanden}

\testReport{15.7.2018 16:40}{Oliver von Seydlitz}{
  \testCancelReservation
}{Termin ist frei, Gruppe 2 hat keine Reservierung mehr}
{Test bestanden}

\testReport{15.7.2018 16:40}{Oliver von Seydlitz}{
  \testCancelOtherReservation
}{Oberfläche lässt Stornierung nicht zu, wenn der Termin durch andere Gruppe reserviert ist}
{Test bestanden}

\testReport{15.7.2018 16:40}{Oliver von Seydlitz}{
  \testCancelReservationNoReservationPresent
}{Oberfläche lässt Stornierung nicht zu, wenn der nicht reserviert ist}
{Test bestanden}

\testReport{15.7.2018 16:45}{Oliver von Seydlitz}{
  \testBook
}{Termin ist durch Gruppe gebucht mit gewünschter Startzeit}
{Test bestanden}

\testReport{15.7.2018 16:50}{Oliver von Seydlitz}{
  \testBookGroupBookedAlready
}{Oberfläche lässt Buchung nicht zu, wenn Gruppe bereits gebucht hat}
{Test bestanden}

\testReport{15.7.2018 16:50}{Oliver von Seydlitz}{
  \testBookGroupHasReservation
}{Termin ist durch Gruppe gebucht, Reservierung ist nicht mehr vorhanden}
{Test bestanden}

\testReport{15.7.2018 16:50}{Oliver von Seydlitz}{
  \testBookReservedByOther
}{Oberfläche lässt Buchung eines durch eine andere Gruppe reservierten Termins nicht zu}
{Test bestanden}

\subsubsection{Abnahmetests Z146b 16.7.2018}

\testReport{16.7.2018 10:20}{Leo Lindhorst und Oliver von Seydlitz}{
  \testGenerate
}{15 freie Termine an den Wochentagen ab 9.7., 10 Gruppen}{Test bestanden}

\testReport{16.7.2018 10:25}{Leo Lindhorst und Oliver von Seydlitz}{
  \testGenerateNotMonday
}{Warnhinweis, da 10.7. kein Montag ist}{Test bestanden}

\testReport{16.7.2018 10:25}{Leo Lindhorst und Oliver von Seydlitz}{
  \testGenerateZeroGroups
}{Warnhinweis, da 0 Gruppen unzulässig sind}{Test bestanden}

\testReport{16.7.2018 10:30}{Leo Lindhorst und Oliver von Seydlitz}{
  \testFreeDeactivated
}{}{Test bestanden}

\testReport{16.7.2018 10:30}{Leo Lindhorst und Oliver von Seydlitz}{
  \testFreeBooked
}{Termin 9.7. hat Zustand frei, ansonsten keine Änderung}{Test bestanden}

\testReport{16.7.2018 10:30}{Leo Lindhorst und Oliver von Seydlitz}{
  \testDeactivateFree
}{Termin 11.7. hat Zustand frei, die Buchung ist nicht mehr vorhanden}{Test bestanden}

\testReport{16.7.2018 10:30}{Leo Lindhorst und Oliver von Seydlitz}{
  \testDeactivateReserved
}{Termin ist deaktiviert, ansonsten keine Änderung}{Test bestanden}

\testReport{16.7.2018 10:30}{Leo Lindhorst und Oliver von Seydlitz}{
  \testSetTimeWindow
}{Termin mit neuem Zeitfenster 08:00-12:00}{Test bestanden}

\testReport{16.7.2018 10:30}{Leo Lindhorst und Oliver von Seydlitz}{
  \testSetTimeWindowInvalid
}{Fehlermeldung}{Test bestanden}

\testReport{16.7.2018 10:30}{Leo Lindhorst und Oliver von Seydlitz}{
  \testSetNote
}{}{Test bestanden}

\testReport{16.7.2018 10:30}{Leo Lindhorst und Oliver von Seydlitz}{
  \testSetBookingRoomTime
}{Raum und Zeit sind gesetzt, der Raum hat aber Leerzeichen anghängt}{Makel, Ursache ist der Datentyp CHAR statt VARCHAR für den Raum. Sollte behoben werden}

\subsubsection{Abnahmetests Z146b 16.7.2018, zweiter Durchlauf}
Folgende Test wurden nach Behebung des falschen Datentyps durchgeführt.

\testReport{16.7.2018 11:20}{Leo Lindhorst und Oliver von Seydlitz}{
  \testGenerate
}{15 freie Termine an den Wochentagen ab 9.7., 10 Gruppen}{Test bestanden}

\testReport{16.7.2018 11:25}{Leo Lindhorst und Oliver von Seydlitz}{
  \testGenerateNotMonday
}{Warnhinweis, da 10.7. kein Montag ist}{Test bestanden}

\testReport{16.7.2018 11:25}{Leo Lindhorst und Oliver von Seydlitz}{
  \testGenerateZeroGroups
}{Warnhinweis, da 0 Gruppen unzulässig sind}{Test bestanden}

\testReport{16.7.2018 11:25}{Leo Lindhorst und Oliver von Seydlitz}{
  \testFreeDeactivated
}{Termin 9.7. hat Zustand frei, ansonsten keine Änderung}{Test bestanden}

\testReport{16.7.2018 11:25}{Leo Lindhorst und Oliver von Seydlitz}{
  \testFreeBooked
}{Termin 11.7. hat Zustand frei, die Buchung ist nicht mehr vorhanden}{Test bestanden}

\testReport{16.7.2018 11:25}{Leo Lindhorst und Oliver von Seydlitz}{
  \testDeactivateFree
}{Termin ist deaktiviert, ansonsten keine Änderung}{Test bestanden}

\testReport{16.7.2018 11:30}{Leo Lindhorst und Oliver von Seydlitz}{
  \testDeactivateReserved
}{Termin ist deaktiviert, die Reservierung ist nicht mehr vorhanden}{Test bestanden}

\testReport{16.7.2018 11:30}{Leo Lindhorst und Oliver von Seydlitz}{
  \testSetTimeWindow
}{Termin mit neuem Zeitfenster 08:00-12:00}{Test bestanden}

\testReport{16.7.2018 11:30}{Leo Lindhorst und Oliver von Seydlitz}{
  \testSetTimeWindowInvalid
}{Fehlermeldung}{Test bestanden}

\testReport{16.7.2018 11:30}{Leo Lindhorst und Oliver von Seydlitz}{
  \testSetNote
}{Termin hat neue Bemerkung "`Hier könnte Ihre Werbung stehen"'}{Test bestanden}

\testReport{16.7.2018 11:30}{Leo Lindhorst und Oliver von Seydlitz}{
  \testSetBookingRoomTime
}{Termin mit Endzeit und Raum wie festgelegt}{Test bestanden}

\testReport{16.7.2018 11:30}{Leo Lindhorst und Oliver von Seydlitz}{
  \testSetBookingRoomTimeNotBooked
}{Oberfläche lässt Festlegung nur zu, wenn Buchung vorhanden ist}{Test bestanden}

\testReport{16.7.2018 11:30}{Leo Lindhorst und Oliver von Seydlitz}{
  \testShowAppointments
}{Übersicht der Termine}{Test bestanden}

\testReport{16.7.2018 11:30}{Leo Lindhorst und Oliver von Seydlitz}{
  \testViewGroups
}{Übersicht der Gruppen}{Test bestanden}

\testReport{16.7.2018 11:30}{Leo Lindhorst und Oliver von Seydlitz}{
  \testDeleteGroup
}{Gruppe ist nicht mehr vorhanden, Buchung am 16.7. nicht mehr vorhanden}{Test bestanden}

\testReport{16.7.2018 11:30}{Leo Lindhorst und Oliver von Seydlitz}{
  \testCreateGroup
}{Neue Gruppe mit Nummer 8 angelegt}{Test bestanden}

\testReport{16.7.2018 11:30}{Leo Lindhorst und Oliver von Seydlitz}{
  \testLogInGroup
}{Terminübersicht}{Test bestanden}

\testReport{16.7.2018 11:30}{Leo Lindhorst und Oliver von Seydlitz}{
  \testReserveDeactivated
}{Oberfläche lässt Reservierung deaktivierter Termine nicht zu}{Test bestanden}

\testReport{16.7.2018 11:30}{Leo Lindhorst und Oliver von Seydlitz}{
  \testReserveBooked
}{Oberfläche lässt Reservierung gebuchter Termine nicht zu}{Test bestanden}

\testReport{16.7.2018 11:30}{Leo Lindhorst und Oliver von Seydlitz}{
  \testReserveGroupAlreadyBooked
}{Oberfläche lässt Reservierung nicht zu, wenn Gruppe bereits gebucht hat}{Test bestanden}

\testReport{16.7.2018 11:35}{Leo Lindhorst und Oliver von Seydlitz}{
  \testCancelReservation
}{Termin ist wieder frei}{Test bestanden}

\testReport{16.7.2018 11:35}{Leo Lindhorst und Oliver von Seydlitz}{
  \testCancelOtherReservation
}{Oberfläche lässt nur Stornierung eigener Reservierungen zu}{Test bestanden}

\testReport{16.7.2018 11:35}{Leo Lindhorst und Oliver von Seydlitz}{
  \testCancelReservationNoReservationPresent
}{Oberfläche lässt Stornierung nur bei vorhandener Reservierung zu}{Test bestanden}

\testReport{16.7.2018 11:35}{Leo Lindhorst und Oliver von Seydlitz}{
  \testBook
}{Termin ist durch Gruppe gebucht mit gewünschter Startzeit}{Test bestanden}

\testReport{16.7.2018 11:35}{Leo Lindhorst und Oliver von Seydlitz}{
  \testBookGroupBookedAlready
}{Oberfläche lässt Buchung nicht zu, wenn Gruppe bereits gebucht hat}{Test bestanden}

\testReport{16.7.2018 11:45}{Leo Lindhorst und Oliver von Seydlitz}{
  \testBookInvalidStartTime
}{Fehlermeldung, da Startzeit nicht in Zeitfenster}{Test bestanden}

\testReport{16.7.2018 11:45}{Leo Lindhorst und Oliver von Seydlitz}{
  \testBookReservedByOther
}{Oberfläche lässt Buchung nicht zu, wenn Termin durch andere Gruppe reserviert ist}{Test bestanden}

\testReport{16.7.2018 11:45}{Leo Lindhorst und Oliver von Seydlitz}{
  \testCreateGroupAndBook
}{Termin ist gebucht}{Test bestanden}
