\documentclass{scrartcl}

\usepackage[ngerman]{babel}
\usepackage{graphicx}

\begin{document}

\section {Projekt Management}
Der folgende Abschnitt dokumentiert Entscheidungen und Planungen des Projektmanagement.

\subsection{Vorgehensmodell}
\begin{itemize}
\item Für das Projekt wurde das Wasserfallmodell mit Rückkopplung aus folgenden Gründen gewählt:
\item Das Wasserfallmodell ist sehr einfach zu erlernen und erzeugt kaum Zusatzaufwände.
\item Es sind keine Anforderungsänderungen zu erwarten, da das Problem wenig komplex und durch den Auftraggeber verstanden ist.
\item Die Problemstellung ist verhältnismäßig klein und einfach verständlich, sodass eine nahezu vollständige bzw. vollständige Anforderungsanalyse verhältnismäßig wahrscheinlich ist.
\item Grundlegende Fehler in den einzelnen Stufen sind aufgrund der kleinen Projekt- und Problemgröße ausreichend unwahrscheinlich, kleine Fehler können durch das Konzept der Rückkopplung korrigiert werden.
\end{itemize}

\begin{figure}
	\includegraphics[width=\textwidth]{Bilder/wasserwrueck}
	\caption{Wasserfallmodell mit Rückkopplungen}
	\label{fig:wasserwrueck}
\end{figure}

\subsection{Meilensteinplan}
Die folgenden Meilensteine wurden für das Projekt definiert:

\begin{table}[h]
	\begin{tabular}{l|l|l}
	Meilenstein1 & Beschreibung & Termin \\
	\hline
	M1 & Das Projektteam ist gebildet & 2. Mai 2018 \\
	\hline
	M2 & Die Anforderungsanalyse ist abgeschlossen & 16. Mai 2018 \\
	\hline
	M3 & Der vorläufige Softwareentwurf ist abgeschlossen & 18. Juni 2018 \\
	\hline
	M4 & Die Implementierung der Software ist abgeschlossen & 6. Juli 2018 \\
	\hline
	M5 & Die Softwaretests wurden durchgeführt & 13. Juli 2018 \\
	\hline
	M6 & Die Software wurde übergeben & 17. Juli 2018 \\
	\hline
	\end{tabular}
	\caption{Meilensteinplan}
	\label{tab:Meilensteinplan}
\end{table}

\begin{figure}
	\includegraphics[width=\textwidth]{Bilder/projektablaufplan}
	\caption{Projektablaufplan}
	\label{fig:projektablaufplan}
\end{figure}


\end{document}
