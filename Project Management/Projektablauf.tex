\documentclass{scrartcl}

\usepackage[ngerman]{babel}
\usepackage{graphicx}
\usepackage[final]{pdfpages}
\setboolean{@twoside}{false}
\usepackage{hyperref}

\usepackage[utf8]{inputenc}

%so that figrues go where they should
\usepackage{float}

\begin{document}

\section{Projekt Management}
Der folgende Abschnitt dokumentiert Entscheidungen und Planungen des Projektmanagements.

\subsection{Projektziel}
Als Grundlage des Projektes wurde im ersten Workshop mit dem Auftraggeber der Projektkontext und das daraus folgende Projektziel definiert.

\subsubsection{Thema}
Das Projekt beschäftigt sich mit der Methode, wie die Prüfungs-Präsentationstermine für den Beleg im Fach Software Engineering 2 der HTW Dresden festgelegt und verwaltet werden.

\subsubsection{Ausgangslage}
Aktuell wird eine Excel-Tabelle für die Organisation genutzt. Diese liegt nur der Prüferin vor. In dieser Tabelle sind die möglichen Zeiten für einen Termin festgehalten. Der Raum, in dem dieser stattfindet ist noch nicht bekannt. Termine und Terminänderungen werden über E-Mail und Telefon in Absprache mit der Prüferin ausgemacht und von ihr in die Liste eingetragen.

\subsubsection{Problem}
Die Terminbelegung für Prüfungspräsentationen über E-Mail und manuelles Editieren der Excel-Tabelle ist zeitaufwändig und kann zu Fehlern führen.

\subsubsection{Projektziel}
Es soll ein Software-System entwickelt werden, über das die Terminplanung geregelt wird. Es soll folgendes gewährleisten:
\begin{itemize}
\item Ein gleichzeitiges Buchen darf nicht möglich sein um Konflikte zu vermeiden.
\item Die Termine müssen vom Prüfer flexibel angegeben werden können.
\item Organisation muss dezentral sein um den Prüfer zu entlasten.
\end{itemize}

\subsection{Risikomanagement}
Für das Projekt wurden die in Tabelle \ref{tab:Projektrisiken} dargestellten Risiken mit entsprechenden Auswirkungen und Minimierungsmaßnahmen identifiziert.
\begin{table}[h]
	\begin{tabular}{p{.3\linewidth}|p{.15\linewidth}|p{.3\linewidth}|p{.15\linewidth}}
	Risiko & Kritikalität\linebreak (1 unkritisch - 3 kritisch) & Maßnahme & Risiko nach Maßnahme\linebreak (1 unkritisch - 3 kritisch)\\
	\hline
	Schwerwiegende Fehler in der Projektumsetzung die kurz vor Projektende entdeckt werden führen zu einem nicht funktionalem Produkt & 3 & Regelmäßige Durchführung von Systemtests, Unit Tests, Durchführung von Code Reviews & 2 \\
	\hline
	Der Ausfall eines Team-Mitgliedes führt zu schwerwiegenden Verzögerungen/Blockaden & 2 & Alle Themen werden von mehreren Mitgliedern bearbeitet/verstanden, Arbeitsartefakte werden möglichst schnell zugänglich gemacht & 1 \\
	\hline
	Auf Erkenntnisse während der Umsetzungsphasen können aufgrund der Festlegung von vorherigen Phasen nicht reagiert werden (Anforderungsänderungen, Missverständnisse, Erkenntnisse während der Implementierung)  & 2 & Nutzung eines Vorgehensmodells mit Rückkopplungen & 1 \\
	\hline
	\end{tabular}
	\caption{Projektrisiken}
	\label{tab:Projektrisiken}
\end{table}

\subsection{Vorgehensmodell}
Für das Projekt wurde sich für ein Vorgehen nach dem "Wasserfallmodell mit Rückkopplungen" entschieden. Das Modell besteht aus sequentiell abgearbeiteten Phasen, deren Ergebnis jeweils Grundlage für die nächste Phase ist. Erkenntnisse aus einer Phase können dabei in die vorausgehende Phase "zurückgekoppelt" werden, um auf Änderungen oder Probleme reagieren zu können. Der Ablauf wird in Abbildung \ref{fig:wasserwrueck} dargestellt. Die folgenden Gründe führten zu der Entscheidung für das "Wasserfallmodell mit Rückkopplungen":
\begin{itemize}
\item Das Wasserfallmodell ist sehr einfach zu erlernen und erzeugt kaum Zusatzaufwände.
\item Es sind keine Anforderungsänderungen zu erwarten, da das Problem wenig komplex und durch den Auftraggeber verstanden ist.
\item Die Problemstellung ist verhältnismäßig klein und einfach verständlich, sodass eine nahezu vollständige bzw. vollständige Anforderungsanalyse verhältnismäßig wahrscheinlich ist.
\item Grundlegende Fehler in den einzelnen Stufen sind aufgrund der kleinen Projekt- und Problemgröße ausreichend unwahrscheinlich, kleine Fehler können durch das Konzept der Rückkopplung korrigiert werden.
\end{itemize}

\begin{figure}[H]
	\includegraphics[width=\textwidth]{Bilder/wasserwrueck}
	\caption{Wasserfallmodell mit Rückkopplungen}
	\label{fig:wasserwrueck}
\end{figure}

\subsection{Projektrollen}
Für das Projekt wurden verschiedene Rollen/Verantwortungsbereiche identifiziert und den Mitgliedern des Projektteams zugeordnet. Die Zuordnung ist in Tabelle \ref{tab:Projektrollen} einsehbar. Die entsprechenden Aufgabengebiete werde von verschiedenen Team-Mitgliedern arbeitsteilig

\begin{table}[h]
	\begin{tabular}{l|l}
	Rolle/Verantwortungsbereich & Team-Mitglied \\
	\hline
	Projektleiter & Leo Lindhorst \\
	\hline
	Analyse & Oliver von Seydlitz \\
	\hline
	Entwurf & Denis Keiling \\
	\hline
	Implementierung & Leo Lindhorst \\
	\hline
	Test & Oliver von Seydlitz \\
	\hline
	Datenbank & Hung Nguyen \\
	\hline
	Dokumentation & Fabian Krehnke \\
	\hline
	Qualitätssicherung & Denis Keiling \\
	\hline
	\end{tabular}
	\caption{Projektrollen}
	\label{tab:Projektrollen}
\end{table}

\subsection{Terminplanung}
\subsubsection{Meilensteinplan}
Als grundlegende Terminstruktur wurden folgende Meilensteine für das Projekt definiert:

\begin{table}[h]
	\begin{tabular}{l|l|l}
	Meilenstein1 & Beschreibung & Termin \\
	\hline
	M1 & Das Projektteam ist gebildet & 2. Mai 2018 \\
	\hline
	M2 & Die Anforderungsanalyse ist abgeschlossen & 16. Mai 2018 \\
	\hline
	M3 & Der vorläufige Softwareentwurf ist abgeschlossen & 18. Juni 2018 \\
	\hline
	M4 & Die Implementierung der Software ist abgeschlossen & 6. Juli 2018 \\
	\hline
	M5 & Die Softwaretests wurden durchgeführt & 13. Juli 2018 \\
	\hline
	M6 & Die Software wurde übergeben & 17. Juli 2018 \\
	\hline
	\end{tabular}
	\caption{Meilensteinplan}
	\label{tab:Meilensteinplan}
\end{table}

\subsection{Team-Organisation}
Um die Team-Arbeit zu organisieren, wurden die folgenden Techniken und Methoden verwendet.
\subsubsection{Aufgabenverwaltung}
Um die ausstehenden Aufgaben, die Verantwortlichen Team-Mitglieder und den Abarbeitungszustand transparent und nachvollziehbar zu machen, wurde ein Aufgabenmanagement-System verwendet. Dazu wurde der Dienst ClickUp (\href{https://clickup.com}{https://clickup.com}) verwendet. Dort können Arbeitspakete als "Tickets" angelegt werden. Team-Mitgliedern die diese Aufgabe übernehmen, kann dieses Paket zugewiesen werden und die können den Abarbeitungszustand auf einem "Aufgaben-Board" sichtbar machen. Dies ist in Abbildung \ref{fig:clickup} exemplarisch dargestellt.

\begin{figure}[H]
	\includegraphics[width=\textwidth]{Bilder/clickup}
	\caption{Aufgaben-Board in ClickUp}
	\label{fig:clickup}
\end{figure}

\subsubsection{Teamkommunikation}
Damit sich alle Mitglieder einfach untereinander abstimmen können, ist es notwendig eine zentrale Kommunikationslösung einzusetzen. Dazu wurde sich auf den Dienst Slack (\href{https://slack.com/}{https://slack.com/}) verwendet. Der Dienst stellt den Team-Mitgliedern Chat-Räume zur Verfügung, sowohl um private und direkt mit einer einzelnen Person zu kommunizieren, als auch um innerhalb einer Interessengruppe zu kommunizieren. Dies ist in Abbildung \ref{fig:slack} zu erkennen.

\begin{figure}[H]
	\includegraphics[width=\textwidth]{Bilder/slack}
	\caption{Chat-Raum in Slack}
	\label{fig:slack}
\end{figure}

\subsubsection{Verteilte Quellcode \& Dokumentenbearbeitung}
Um Arbeitsteilig am Produkt und den Projekt-Dokumenten arbeiten zu können, benötigt es eine zentrale Plattform um diese Daten zu verwalten und zu bearbeiten. Dazu wurde für auf die Verwendung von Git als Versionskontroll-System, sowie den Dienst GitHub (\href{https://github.com/}{https://github.com/}) als zentrale Quellcodeverwaltungsplattform entschieden. Sowohl der Quellcode des Produktes, als auch die Projektdokumente in entsprechenden Text-Formaten (Markdown oder Latex) werden in verschiedenen GitHub Repositories verwaltet und können dort von allen Team-Mitgliedern bearbeitet und begutachtet werden. Um die Qualitätsanforderungen sicherzustellen, ist es notwendig Änderungen an den Projektdokumenten und dem Quellcode einer Prüfung durch andere Team-Mitglieder zu unterziehen. Der Pull Request Mechanismus von GitHub bietet dafür einen einfachen und effizienten Weg. Dies ist in Abbildung \ref{fig:github} exemplarisch dargestellt.

\begin{figure}[H]
	\includegraphics[width=\textwidth]{Bilder/github}
	\caption{Pull Request in GitHub}
	\label{fig:github}
\end{figure}

\subsubsection{Projektablaufplan}
Der Projektablaufplan, sowie der Ressourcenplan wurde nicht zu Beginn des Projektes, sondern erst am 18.06.2018 erstellt. Zum Zwecke der Dokumentation und um die ausstehenden Arbeiten in einen Kontext setzen zu können, wurden die bis zu diesem Zeitpunkt geschehenen Arbeiten anhand bestehender Arbeitsartefakte und Aufzeichnungen rekonstruiert und in den Projektablaufplan eingetragen. Ab dem 18.06.2018 wurde dann ein Projektablaufplan und ein Ressourcenplan als Gantt-Chart wie in Abbildung \ref{fig:projektablaufplan} erstellt. Der vollständige Projektablaufplan, sowie Ressourcenplan befindet sich im Anhang unter Abschnitt \ref{sec:projektablaufplan}.
\begin{figure}[H]
	\includegraphics[width=\textwidth]{Bilder/projektablaufplan}
	\caption{Projektablaufplan}
	\label{fig:projektablaufplan}
\end{figure}

\section{Anhang}
\subsection{Projektablaufplan}\label{sec:projektablaufplan}
\includepdf[pages=-, angle=90]{projektablaufplan}

\end{document}
